%% Generated by Sphinx.
\def\sphinxdocclass{report}
\documentclass[letterpaper,10pt,english]{sphinxmanual}
\ifdefined\pdfpxdimen
   \let\sphinxpxdimen\pdfpxdimen\else\newdimen\sphinxpxdimen
\fi \sphinxpxdimen=.75bp\relax
\ifdefined\pdfimageresolution
    \pdfimageresolution= \numexpr \dimexpr1in\relax/\sphinxpxdimen\relax
\fi
%% let collapsible pdf bookmarks panel have high depth per default
\PassOptionsToPackage{bookmarksdepth=5}{hyperref}

\PassOptionsToPackage{booktabs}{sphinx}
\PassOptionsToPackage{colorrows}{sphinx}

\PassOptionsToPackage{warn}{textcomp}
\usepackage[utf8]{inputenc}
\ifdefined\DeclareUnicodeCharacter
% support both utf8 and utf8x syntaxes
  \ifdefined\DeclareUnicodeCharacterAsOptional
    \def\sphinxDUC#1{\DeclareUnicodeCharacter{"#1}}
  \else
    \let\sphinxDUC\DeclareUnicodeCharacter
  \fi
  \sphinxDUC{00A0}{\nobreakspace}
  \sphinxDUC{2500}{\sphinxunichar{2500}}
  \sphinxDUC{2502}{\sphinxunichar{2502}}
  \sphinxDUC{2514}{\sphinxunichar{2514}}
  \sphinxDUC{251C}{\sphinxunichar{251C}}
  \sphinxDUC{2572}{\textbackslash}
\fi
\usepackage{cmap}
\usepackage[T1]{fontenc}
\usepackage{amsmath,amssymb,amstext}
\usepackage{babel}



\usepackage{tgtermes}
\usepackage{tgheros}
\renewcommand{\ttdefault}{txtt}



\usepackage[Sonny]{fncychap}
\ChNameVar{\Large\normalfont\sffamily}
\ChTitleVar{\Large\normalfont\sffamily}
\usepackage{sphinx}

\fvset{fontsize=auto}
\usepackage{geometry}


% Include hyperref last.
\usepackage{hyperref}
% Fix anchor placement for figures with captions.
\usepackage{hypcap}% it must be loaded after hyperref.
% Set up styles of URL: it should be placed after hyperref.
\urlstyle{same}

\addto\captionsenglish{\renewcommand{\contentsname}{Contents:}}

\usepackage{sphinxmessages}
\setcounter{tocdepth}{1}



\title{maravilla}
\date{May 07, 2025}
\release{0.6.0}
\author{Carlo\sphinxhyphen{}Ariel Vanegas}
\newcommand{\sphinxlogo}{\vbox{}}
\renewcommand{\releasename}{Release}
\makeindex
\begin{document}

\ifdefined\shorthandoff
  \ifnum\catcode`\=\string=\active\shorthandoff{=}\fi
  \ifnum\catcode`\"=\active\shorthandoff{"}\fi
\fi

\pagestyle{empty}
\sphinxmaketitle
\pagestyle{plain}
\sphinxtableofcontents
\pagestyle{normal}
\phantomsection\label{\detokenize{index::doc}}


\sphinxAtStartPar
Add your content using \sphinxcode{\sphinxupquote{reStructuredText}} syntax. See the
\sphinxhref{https://www.sphinx-doc.org/en/master/usage/restructuredtext/index.html}{reStructuredText}
documentation for details.
\index{module@\spxentry{module}!num@\spxentry{num}}\index{num@\spxentry{num}!module@\spxentry{module}}\index{totient() (in module num)@\spxentry{totient()}\spxextra{in module num}}\phantomsection\label{\detokenize{index:module-num}}

\begin{fulllineitems}
\phantomsection\label{\detokenize{index:num.totient}}
\pysigstartsignatures
\pysiglinewithargsret
{\sphinxcode{\sphinxupquote{num.}}\sphinxbfcode{\sphinxupquote{totient}}}
{\sphinxparam{\DUrole{n}{x}\DUrole{p}{:}\DUrole{w}{ }\DUrole{n}{int}}}
{{ $\rightarrow$ int}}
\pysigstopsignatures
\sphinxAtStartPar
fonction phi d\textquotesingle{}euler

\end{fulllineitems}

\index{module@\spxentry{module}!limit@\spxentry{limit}}\index{limit@\spxentry{limit}!module@\spxentry{module}}\index{fourier\_transform() (in module limit)@\spxentry{fourier\_transform()}\spxextra{in module limit}}\phantomsection\label{\detokenize{index:module-limit}}

\begin{fulllineitems}
\phantomsection\label{\detokenize{index:limit.fourier_transform}}
\pysigstartsignatures
\pysiglinewithargsret
{\sphinxcode{\sphinxupquote{limit.}}\sphinxbfcode{\sphinxupquote{fourier\_transform}}}
{\sphinxparam{\DUrole{n}{func}\DUrole{p}{:}\DUrole{w}{ }\DUrole{n}{Callable\DUrole{p}{{[}}\DUrole{p}{{[}}float\DUrole{p}{{]}}\DUrole{p}{,}\DUrole{w}{ }float\DUrole{p}{{]}}}}\sphinxparamcomma \sphinxparam{\DUrole{n}{delta}\DUrole{p}{:}\DUrole{w}{ }\DUrole{n}{float}\DUrole{w}{ }\DUrole{o}{=}\DUrole{w}{ }\DUrole{default_value}{0.001}}\sphinxparamcomma \sphinxparam{\DUrole{n}{n}\DUrole{p}{:}\DUrole{w}{ }\DUrole{n}{int}\DUrole{w}{ }\DUrole{o}{=}\DUrole{w}{ }\DUrole{default_value}{1000}}}
{{ $\rightarrow$ Callable\DUrole{p}{{[}}\DUrole{p}{{[}}float\DUrole{p}{{]}}\DUrole{p}{,}\DUrole{w}{ }complex\DUrole{p}{{]}}}}
\pysigstopsignatures
\sphinxAtStartPar
transformée de fourier de f

\end{fulllineitems}

\index{integral() (in module limit)@\spxentry{integral()}\spxextra{in module limit}}

\begin{fulllineitems}
\phantomsection\label{\detokenize{index:limit.integral}}
\pysigstartsignatures
\pysiglinewithargsret
{\sphinxcode{\sphinxupquote{limit.}}\sphinxbfcode{\sphinxupquote{integral}}}
{\sphinxparam{\DUrole{n}{func}\DUrole{p}{:}\DUrole{w}{ }\DUrole{n}{Callable\DUrole{p}{{[}}\DUrole{p}{{[}}float\DUrole{p}{{]}}\DUrole{p}{,}\DUrole{w}{ }complex\DUrole{p}{{]}}}}\sphinxparamcomma \sphinxparam{\DUrole{n}{a}\DUrole{p}{:}\DUrole{w}{ }\DUrole{n}{float}}\sphinxparamcomma \sphinxparam{\DUrole{n}{b}\DUrole{p}{:}\DUrole{w}{ }\DUrole{n}{float}}\sphinxparamcomma \sphinxparam{\DUrole{n}{delta}\DUrole{p}{:}\DUrole{w}{ }\DUrole{n}{float}\DUrole{w}{ }\DUrole{o}{=}\DUrole{w}{ }\DUrole{default_value}{0.001}}}
{{ $\rightarrow$ complex}}
\pysigstopsignatures
\sphinxAtStartPar
intègre une fonction entre a et b

\end{fulllineitems}

\index{sum() (in module limit)@\spxentry{sum()}\spxextra{in module limit}}

\begin{fulllineitems}
\phantomsection\label{\detokenize{index:limit.sum}}
\pysigstartsignatures
\pysiglinewithargsret
{\sphinxcode{\sphinxupquote{limit.}}\sphinxbfcode{\sphinxupquote{sum}}}
{\sphinxparam{\DUrole{n}{func}\DUrole{p}{:}\DUrole{w}{ }\DUrole{n}{Callable\DUrole{p}{{[}}\DUrole{p}{{[}}complex\DUrole{p}{{]}}\DUrole{p}{,}\DUrole{w}{ }complex\DUrole{p}{{]}}}}\sphinxparamcomma \sphinxparam{\DUrole{n}{a}\DUrole{p}{:}\DUrole{w}{ }\DUrole{n}{int}}\sphinxparamcomma \sphinxparam{\DUrole{n}{b}\DUrole{p}{:}\DUrole{w}{ }\DUrole{n}{int}}}
{{ $\rightarrow$ complex}}
\pysigstopsignatures
\sphinxAtStartPar
prodède à la somme d\textquotesingle{}une fonction entre a et b

\end{fulllineitems}

\index{module@\spxentry{module}!polynomial@\spxentry{polynomial}}\index{polynomial@\spxentry{polynomial}!module@\spxentry{module}}\index{evaluer() (in module polynomial)@\spxentry{evaluer()}\spxextra{in module polynomial}}\phantomsection\label{\detokenize{index:module-polynomial}}

\begin{fulllineitems}
\phantomsection\label{\detokenize{index:polynomial.evaluer}}
\pysigstartsignatures
\pysiglinewithargsret
{\sphinxcode{\sphinxupquote{polynomial.}}\sphinxbfcode{\sphinxupquote{evaluer}}}
{\sphinxparam{\DUrole{n}{polynome}\DUrole{p}{:}\DUrole{w}{ }\DUrole{n}{list}}\sphinxparamcomma \sphinxparam{\DUrole{n}{x}\DUrole{p}{:}\DUrole{w}{ }\DUrole{n}{int}}}
{{ $\rightarrow$ int}}
\pysigstopsignatures
\sphinxAtStartPar
évalue le polynôme en x,où le terme polynome{[}i{]} est associé à la puissance x\textasciicircum{}i

\end{fulllineitems}

\index{isIrreductibleMod() (in module polynomial)@\spxentry{isIrreductibleMod()}\spxextra{in module polynomial}}

\begin{fulllineitems}
\phantomsection\label{\detokenize{index:polynomial.isIrreductibleMod}}
\pysigstartsignatures
\pysiglinewithargsret
{\sphinxcode{\sphinxupquote{polynomial.}}\sphinxbfcode{\sphinxupquote{isIrreductibleMod}}}
{\sphinxparam{\DUrole{n}{polynome}\DUrole{p}{:}\DUrole{w}{ }\DUrole{n}{list}}\sphinxparamcomma \sphinxparam{\DUrole{n}{mod}\DUrole{p}{:}\DUrole{w}{ }\DUrole{n}{int}}}
{{ $\rightarrow$ bool}}
\pysigstopsignatures
\sphinxAtStartPar
tente de trouver un facteur de degré 1 dans un polynôme (mod n)
Retourne True si le polynôme est irréductible

\end{fulllineitems}

\index{isIrreductibleZ() (in module polynomial)@\spxentry{isIrreductibleZ()}\spxextra{in module polynomial}}

\begin{fulllineitems}
\phantomsection\label{\detokenize{index:polynomial.isIrreductibleZ}}
\pysigstartsignatures
\pysiglinewithargsret
{\sphinxcode{\sphinxupquote{polynomial.}}\sphinxbfcode{\sphinxupquote{isIrreductibleZ}}}
{\sphinxparam{\DUrole{n}{polynome}\DUrole{p}{:}\DUrole{w}{ }\DUrole{n}{list}}}
{{ $\rightarrow$ bool}}
\pysigstopsignatures
\sphinxAtStartPar
tente de trouver un facteur de degré 1 à un polynôme (dans les entiers)
Retourne True si le polynôme est irréductible

\end{fulllineitems}



\renewcommand{\indexname}{Python Module Index}
\begin{sphinxtheindex}
\let\bigletter\sphinxstyleindexlettergroup
\bigletter{l}
\item\relax\sphinxstyleindexentry{limit}\sphinxstyleindexpageref{index:\detokenize{module-limit}}
\indexspace
\bigletter{n}
\item\relax\sphinxstyleindexentry{num}\sphinxstyleindexpageref{index:\detokenize{module-num}}
\indexspace
\bigletter{p}
\item\relax\sphinxstyleindexentry{polynomial}\sphinxstyleindexpageref{index:\detokenize{module-polynomial}}
\end{sphinxtheindex}

\renewcommand{\indexname}{Index}
\printindex
\end{document}